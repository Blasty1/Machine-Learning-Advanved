\documentclass[a4paper]{article}

\usepackage{amsmath}
\usepackage{amsfonts}
\usepackage{xcolor}

\title{E1 - Assignment}
\author{
Bruno Carchia \\
\texttt{carchia@kth.se}
\and
Riccardo Alfonso Cerrone \\
  \texttt{cerrone@kth.se}
}

\begin{document}
\maketitle

\section*{1.2 Inference in Bayesian Networks}
\subsection*{1.2.7 Classify the fish as salmon or sea bass}
Suppose the fish was caught on December 20, with
\[
\mathbb{P}(X_1) = (0.5, 0, 0, 0.5)
\]
corresponding to the four seasons (winter, spring, summer, autumn). 
The lightness has not been measured, but it is known that the fish is thin.

We want to compute the posterior distributions
\[
\mathbb{P}(X_2 \mid X_4 = \text{thin})
\]
in order to classify the fish as salmon or sea bass.

Using the Bayes rule, we have
\[
\mathbb{P}(X_2 \mid X_4 = \text{thin})= \dfrac{\mathbb{P}( X_4 = \text{thin} \mid X_2) \mathbb{P}(X_2)}{\mathbb{P}(X_4 = \text{thin})}
\]
Where:
\begin{itemize}
    \item \begin{math}\mathbb{P}( X_4 = \text{thin} \mid X_2)\end{math} is given by the corresponding CPT
    \item \begin{math}\mathbb{P}(X_2)\end{math} = \(\sum_{x_1}\mathbb{P}(X_2, X_1=x_1)\) = \(\sum_{x_1}\mathbb{P}(X_2 \mid X_1=x_1)\mathbb{P}(X_1=x_1)\) can be computed using the given CPT and the prior on \begin{math}X_1\end{math}
    \item \begin{math}\mathbb{P}(X_4 = \text{thin})\end{math} is a normalizing constant and for this reason we can ignore it in the computation.
\end{itemize}
At the end, we have:
\[
  \mathbb{P}(X_2 | X_4 = \text{thin}) \propto \mathbb{P}(X_4=\text{thin} | X_2) \sum_{x_1}\mathbb{P}(X_2 \mid X_1=x_1)\mathbb{P}(X_1=x_1)
\]
The second step is to compute the above expression for both values of \begin{math}X_2\end{math} (salmon and sea bass) and then compare the results to classify the fish.

\begin{align*} \mathbb{P}(X_2 = \text{salmon} \mid X_4 = \text{thin}) & \propto 0.65 * (0.5*0.88 + 0*0.32 + 0*0.42 + 0.5*0.78) \\ & \propto 0.65 * (0.44 + 0 + 0 + 0.39) \\ & \propto 0.65 * 0.83 \\ & \propto 0.5395 \end{align*}

\begin{align*} \mathbb{P}(X_2 = \text{sea bass} \mid X_4 = \text{thin}) & \propto 0.06 * (0.5*0.12 + 0*0.68 + 0*0.58 + 0.5*0.22) \\ & \propto 0.06 * (0.06 + 0 + 0 + 0.11) \\ & \propto 0.65 * 0.17 \\ & \propto 0.0102 \end{align*}

For the classification we compare the two results:
\[ 0.5395 > 0.0102
\]
so we classify the fish as \textbf{salmon}.
\newpage

\begin{align*}
\mathbb{P}(X_1=x_1 \big | X_3 = \text{medium}, X_4 = \text{thin}) &= \dfrac{\mathbb{P}(X_1=x_1, X_3 = \text{medium}, X_4 = \text{thin})}{\mathbb{P}(X_3 = \text{medium})\mathbb{P}(X_4 = \text{thin} \big | X_3 = \text{medium})}
\\
& \propto \mathbb{P}(X_1=x_1, X_3 = \text{medium}, X_4 = \text{thin})
\\
&= \mathbb{P}(X_1 = x_1) \textcolor{red}{\mathbb{P}(X_3= \text{medium} \big | X_1 = x_1)} \\
& \textcolor{blue}{\mathbb{P}(X_4 = \text{thin} \big | X_3 = \text{medium}, X_1 = x_1)}
\end{align*}

\begin{align*}
\textcolor{blue}{\mathbb{P}(X_4 = \text{T} \big | X_3 = \text{M}, X_1 = x_1)} &= \sum_{x_2} \mathbb{P}(X_4= \text{T}, X_2 = x_2 \big |X_3 = \text{M}, X_1 = x_1) 
\\
&= \sum_{x_2} \mathbb{P}(X_4= \text{T}, X_2 = x_2 \big |X_3 = \text{M}, X_1 = x_1) 
\end{align*}


\end{document}